\documentclass[a4paper,12pt]{article}

\usepackage[hidelinks]{hyperref}
\usepackage{amsmath}
\usepackage{mathtools}
\usepackage{shorttoc}
\usepackage{cmap}
\usepackage[T2A]{fontenc}
\usepackage[utf8]{inputenc}
\usepackage[english, russian]{babel}
\usepackage{xcolor}
\usepackage{graphicx}
\usepackage{float}
\graphicspath{{./img/}}

\definecolor{linkcolor}{HTML}{000000}
\definecolor{urlcolor}{HTML}{0085FF}
\hypersetup{pdfstartview=FitH,  linkcolor=linkcolor,urlcolor=urlcolor, colorlinks=true}

\DeclarePairedDelimiter{\floor}{\lfloor}{\rfloor}

\renewcommand*\contentsname{Содержание}

\newcommand{\plot}[3]{
    \begin{figure}[H]
        \begin{center}
            \includegraphics[scale=0.6]{#1}
            \caption{#2}
            \label{#3}
        \end{center}
    \end{figure}
}

\begin{document}
    \begin{titlepage}
	\begin{center}
		{\large Санкт-Петербургский политехнический университет\\Петра Великого\\}
	\end{center}
	
	\begin{center}
		{\large Физико-механический иститут}
	\end{center}
	
	
	\begin{center}
		{\large Кафедра «Прикладная математика»}
	\end{center}
	
	\vspace{8em}
	
	\begin{center}
		{\bfseries Отчёт по лабораторной работе \\по дисциплине «Компьютерные сети» \\Задача византийских генералов }
	\end{center}
	
	\vspace{4em}
	
	\begin{flushleft}
		\hspace{16em}Выполнил студент:\\\hspace{16em}Куксенко Кирилл Сергеевич\\\hspace{16em}группа: 5040102/20201
		
		\vspace{2em}
		
		\hspace{16em}Проверил:\\\hspace{16em}к.ф.-м.н., доцент\\\hspace{16em}Баженов Александр Николаевич
		
	\end{flushleft}
	
	
	\vspace{6em}
	
	
	\begin{center}
		Санкт-Петербург\\2023 г.
	\end{center}	
	
\end{titlepage}
    \newpage

    \tableofcontents
    \listoffigures
    \newpage

    \section{Постановка задачи}
    \quad Нужно реализовать протокол маршрутизации OSPF (Open Shortest Path First).
    И проверить его работоспособность на следующих видах топологий сети:
    линейная, кольцевидная и звёздная.

    \section{Теория}
    OSPF (Open Shortest Path First) — протокол динамической маршрутизации,
    основанный на технологии отслеживания состояния канала
    и использующий для нахождения кратчайшего пути алгоритм Дейкстры.

    Описание работы протокола.
    \begin{itemize}
        \item После включения маршрутизаторов протокол ищет непосредственно подключенных соседей
        и устанавливает с ними «дружеские» отношения.
        \item Затем они обмениваются друг с другом информацией о подключенных и доступных им сетях.
        То есть они строят карту сети (топологию сети).
        Данная карта одинакова на всех маршрутизаторах.
        \item На основе полученной информации запускается алгоритм SPF (Shortest Path First),
        который рассчитывает оптимальный маршрут к каждой сети.
        Данный процесс похож на построение дерева, корнем которого является сам маршрутизатор,
        а ветвями — пути к доступным сетям.
    \end{itemize}

    \section{Реализация}
    \quad Весь код написан на языке Python (версии 3.7.3).
    Для каждого протокола получатель и отправитель работают параллельно в отдельных потоках.
    \href{https://github.com/kirillkuks/Networks/tree/master/lab2}{Ссылка на GitHub с исходным кодом}.

    \section{Результаты}
    Сначала посмотрим на работу протокола на сети с линейной топологией.
    Узлы сети имеют следующее расположение.
    \plot{full_line_points}{Расположение узлов сети с линейной топологией}{p:fullLinePoints}

    Построим граф сети, указав радиус соединения равным $ r = 1.5 $.
    \plot{full_line}{Граф сети с линейной топологией}{p:fullLine}

    Найдём кратчайшие пути между всеми парами узлов сети.
    Приведём некоторые примеры (полные результаты в файле \textsl{lab2/results/line\_full.txt}).
    \begin{itemize}
        \item Начальный узел $ 0 $
        \begin{itemize}
            \item path 0 -> 1: [0, 1]
            \item path 0 -> 2: [0, 1, 2]
            \item path 0 -> 3: [0, 1, 2, 3]
            \item path 0 -> 4: [0, 1, 2, 3, 4]
            \item path 0 -> 5: [0, 1, 2, 3, 4, 5]
        \end{itemize} 
        \item Начальный узел $ 4 $
        \begin{itemize}
            \item path 4 -> 0: [4, 3, 2, 1, 0]
            \item path 4 -> 1: [4, 3, 2, 1]
            \item path 4 -> 2: [4, 3, 2]
            \item path 4 -> 3: [4, 3]
            \item path 4 -> 5: [4, 5]
        \end{itemize} 
    \end{itemize}

    Теперь уберём из сети узел $ 3 $ (перенеся его достаточно далеко)
    и перестроим граф сети.
    \plot{rm_line}{Граф сети с линейной топологией без $ 3 $ узла}{p:rmLine}

    Приведём кратчайшие пути для тех же пар узлов (полные результаты в файле \textsl{lab2/results/line\_remove.txt}).
    \begin{itemize}
        \item Начальный узел $ 0 $
        \begin{itemize}
            \item path 0 -> 1: [0, 1]
            \item path 0 -> 2: [0, 1, 2]
            \item path 0 -> 3: []
            \item path 0 -> 4: []
            \item path 0 -> 5: []
        \end{itemize} 
        \item Начальный узел $ 4 $
        \begin{itemize}
            \item path 4 -> 0: []
            \item path 4 -> 1: []
            \item path 4 -> 2: []
            \item path 4 -> 3: []
            \item path 4 -> 5: [4, 5]
        \end{itemize} 
    \end{itemize}

    Проведём аналогичную процедуру для сети с кольцевидной топологией.
    \plot{full_ring_points}{Расположение узлов сети с кольцевидной топологией}{p:fullRingPoints}

    Граф, построенный с радиусом соединения $ r = 1.7 $, сети имеет вид.
    \plot{full_ring}{Граф сети с кольцевидной топологией}{p:fullRing}

    Примеры кратчайших путей (подробнее \textsl{lab2/results/ring\_full.txt})
    \begin{itemize}
        \item Начальный узел $ 5 $
        \begin{itemize}
            \item path 5 -> 0: [5, 3, 1, 0]
            \item path 5 -> 1: [5, 3, 1]
            \item path 5 -> 2: [5, 3, 1, 0, 2]
            \item path 5 -> 3: [5, 3]
            \item path 5 -> 4: [5, 3, 1, 0, 2, 4]
            \item path 5 -> 6: [5, 3, 1, 0, 2, 4, 6]
            \item path 5 -> 7: [5, 7]
            \item path 5 -> 8: [5, 3, 1, 0, 2, 4, 6, 8]
            \item path 5 -> 9: [5, 7, 14, 12, 10, 9]
            \item path 5 -> 10: [5, 7, 14, 12, 10]
            \item path 5 -> 11: [5, 7, 14, 12, 10, 9, 11]
            \item path 5 -> 12: [5, 7, 14, 12]
            \item path 5 -> 13: [5, 7, 14, 12, 10, 9, 11, 13]
            \item path 5 -> 14: [5, 7, 14]
            \item path 5 -> 15: [5, 7, 14, 12, 10, 9, 11, 13, 15]
        \end{itemize}
        \item Начальный узел $ 12 $
        \begin{itemize}
            \item path 12 -> 0: [12, 14, 7, 5, 3, 1, 0]
            \item path 12 -> 1: [12, 14, 7, 5, 3, 1]
            \item path 12 -> 2: [12, 14, 7, 5, 3, 1, 0, 2]
            \item path 12 -> 3: [12, 14, 7, 5, 3]
            \item path 12 -> 4: [12, 14, 7, 5, 3, 1, 0, 2, 4]
            \item path 12 -> 5: [12, 14, 7, 5]
            \item path 12 -> 6: [12, 10, 9, 11, 13, 15, 8, 6]
            \item path 12 -> 7: [12, 14, 7]
            \item path 12 -> 8: [12, 10, 9, 11, 13, 15, 8]
            \item path 12 -> 9: [12, 10, 9]
            \item path 12 -> 10: [12, 10]
            \item path 12 -> 11: [12, 10, 9, 11]
            \item path 12 -> 13: [12, 10, 9, 11, 13]
            \item path 12 -> 14: [12, 14]
            \item path 12 -> 15: [12, 10, 9, 11, 13, 15]
        \end{itemize}
    \end{itemize}

    После удаления узла $ 11 $ граф сети имеет вид.
    \plot{rm_ring}{Граф сети с кольцевидной топологией без $ 11 $ узла}{p:rmRing}

    Примеры путей для тех же пар узлов (подробнее \textsl{lab2/results/ring\_remove.txt})
    \begin{itemize}
        \item Начальный узел $ 5 $
        \begin{itemize}
            \item path 5 -> 0: [5, 3, 1, 0]
            \item path 5 -> 1: [5, 3, 1]
            \item path 5 -> 2: [5, 3, 1, 0, 2]
            \item path 5 -> 3: [5, 3]
            \item path 5 -> 4: [5, 3, 1, 0, 2, 4]
            \item path 5 -> 6: [5, 3, 1, 0, 2, 4, 6]
            \item path 5 -> 7: [5, 7]
            \item path 5 -> 8: [5, 3, 1, 0, 2, 4, 6, 8]
            \item path 5 -> 9: [5, 7, 14, 12, 10, 9]
            \item path 5 -> 10: [5, 7, 14, 12, 10]
            \item path 5 -> 11: []
            \item path 5 -> 12: [5, 7, 14, 12]
            \item path 5 -> 13: [5, 3, 1, 0, 2, 4, 6, 8, 15, 13]
            \item path 5 -> 14: [5, 7, 14]
            \item path 5 -> 15: [5, 3, 1, 0, 2, 4, 6, 8, 15]
        \end{itemize}
        \item Начальный узел $ 12 $
        \begin{itemize}
            \item path 12 -> 0: [12, 14, 7, 5, 3, 1, 0]
            \item path 12 -> 1: [12, 14, 7, 5, 3, 1]
            \item path 12 -> 2: [12, 14, 7, 5, 3, 1, 0, 2]
            \item path 12 -> 3: [12, 14, 7, 5, 3]
            \item path 12 -> 4: [12, 14, 7, 5, 3, 1, 0, 2, 4]
            \item path 12 -> 5: [12, 14, 7, 5]
            \item path 12 -> 6: [12, 14, 7, 5, 3, 1, 0, 2, 4, 6]
            \item path 12 -> 7: [12, 14, 7]
            \item path 12 -> 8: [12, 14, 7, 5, 3, 1, 0, 2, 4, 6, 8]
            \item path 12 -> 9: [12, 10, 9]
            \item path 12 -> 10: [12, 10]
            \item path 12 -> 11: []
            \item path 12 -> 13: [12, 14, 7, 5, 3, 1, 0, 2, 4, 6, 8, 15, 13]
            \item path 12 -> 14: [12, 14]
            \item path 12 -> 15: [12, 14, 7, 5, 3, 1, 0, 2, 4, 6, 8, 15]
        \end{itemize}
    \end{itemize}

    Узлы сети со звёздной топологией и центральным узлом $ 0 $ имеют следующее расположение.
    \plot{full_star_points}{Расположение узлов сети с звёздной топологией}{p:fullStarPoints}

    Граф для данной сети имеет вид.
    \plot{full_star}{Граф сети с звёздной топологией}{p:fullStar}

    Некоторые примеры кратчайший путей (подробнее \textsl{lab2/results/star\_full.txt}).
    \begin{itemize}
        \item Начальный узел $ 0 $
        \begin{itemize}
            \item path 0 -> 1: [0, 1]
            \item path 0 -> 2: [0, 2]
            \item path 0 -> 3: [0, 3]
            \item path 0 -> 4: [0, 4]
            \item path 0 -> 5: [0, 5]
            \item path 0 -> 6: [0, 6]
            \item path 0 -> 7: [0, 7]
            \item path 0 -> 8: [0, 8]
        \end{itemize}
        \item Начальный узел $ 7 $
        \begin{itemize}
            \item path 7 -> 0: [7, 0]
            \item path 7 -> 1: [7, 0, 1]
            \item path 7 -> 2: [7, 0, 2]
            \item path 7 -> 3: [7, 0, 3]
            \item path 7 -> 4: [7, 0, 4]
            \item path 7 -> 5: [7, 0, 5]
            \item path 7 -> 6: [7, 0, 6]
            \item path 7 -> 8: [7, 0, 8]
        \end{itemize}
    \end{itemize}

    После удаления центрального узла $ 0 $ граф сети имеет вид.
    \plot{rm_star}{Граф сети с звёздной топологией без центрального узла $ 0 $}{p:rmStar}

    Путей для тех же пар узлов (подробнее \textsl{lab2/results/star\_remove.txt})
    не будет существовать.

    \section{Обсуждение}
    Из полученных результатов можно заметить следующее.
    Сеть с линейной топологией наиболее чувствительна к потерям
    узлов сети, потеря одного узла ведёт к появлению недостижимых узлов.
    Сеть с кольцевидной топологией менее чувствительна к потерям узлов,
    при потере одного узла она переходит в сеть с линейной топологией.
    Сеть со звёздной топологией наименее чувствительна к потере узлов до тех пор,
    пока это не центральный узел.
    В случае потери центрального узла любая пара других узлов становится недостижима. 

\end{document}