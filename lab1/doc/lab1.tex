\documentclass[a4paper,12pt]{article}

\usepackage[hidelinks]{hyperref}
\usepackage{amsmath}
\usepackage{mathtools}
\usepackage{shorttoc}
\usepackage{cmap}
\usepackage[T2A]{fontenc}
\usepackage[utf8]{inputenc}
\usepackage[english, russian]{babel}
\usepackage{xcolor}
\usepackage{graphicx}
\usepackage{float}
\graphicspath{{./img/}}

\definecolor{linkcolor}{HTML}{000000}
\definecolor{urlcolor}{HTML}{0085FF}
\hypersetup{pdfstartview=FitH,  linkcolor=linkcolor,urlcolor=urlcolor, colorlinks=true}

\DeclarePairedDelimiter{\floor}{\lfloor}{\rfloor}

\renewcommand*\contentsname{Содержание}

\newcommand{\plot}[3]{
    \begin{figure}[H]
        \begin{center}
            \includegraphics[scale=0.6]{#1}
            \caption{#2}
            \label{#3}
        \end{center}
    \end{figure}
}

\begin{document}
    \begin{titlepage}
	\begin{center}
		{\large Санкт-Петербургский политехнический университет\\Петра Великого\\}
	\end{center}
	
	\begin{center}
		{\large Физико-механический иститут}
	\end{center}
	
	
	\begin{center}
		{\large Кафедра «Прикладная математика»}
	\end{center}
	
	\vspace{8em}
	
	\begin{center}
		{\bfseries Отчёт по лабораторной работе \\по дисциплине «Компьютерные сети» \\Реализация протокола маршрутизации Open Shortest Path First }
	\end{center}
	
	\vspace{4em}
	
	\begin{flushleft}
		\hspace{16em}Выполнил студент:\\\hspace{16em}Куксенко Кирилл Сергеевич\\\hspace{16em}группа: 5040102/20201
		
		\vspace{2em}
		
		\hspace{16em}Проверил:\\\hspace{16em}к.ф.-м.н., доцент\\\hspace{16em}Баженов Александр Николаевич
		
	\end{flushleft}
	
	
	\vspace{6em}
	
	
	\begin{center}
		Санкт-Петербург\\2023 г.
	\end{center}	
	
\end{titlepage}
    \newpage

    \tableofcontents
    \listoffigures
    \newpage

    \section{Постановка задачи}
    \quad Нужно реализовать систему из двух объектов - отправителя (Sender) и получателя (Receiver),
    которые будут обмениваться сообщениями по каналу связи с помощью протоколов автоматического запроса повторной передачи
    со скользящего окном: Go-Back-N и Selective Repeat.

    Необходимо выяснить зависимость времени работы и количество посланных сообщений от размера плавающего окна 
    и вероятности потери сообщения для каждого протокола и сравнить друг с другом.

    \section{Теория} \label{s:theory}
    \quad Протоколы Go-Back-N и Selective Repeat являются протоколами скользящего окна.
    Основное различие между этими двумя протоколами заключается в том, что после обнаружения подозрительного или поврежденного сообщения
    протокол Go-Back-N повторно передает все сообщения, не получившие подтверждения о получении,
    тогда как протокол Selective Repeat повторно передает только то сообщение, которое оказалось повреждено.

    \section{Реализация}
    \quad Весь код написан на языке Python (версии 3.7.3).
    Для каждого протокола получатель и отправитель работают параллельно в отдельных потоках.
    \href{https://github.com/kirillkuks/Networks/tree/master/lab1}{Ссылка на GitHub с исходным кодом}.

    \section{Результаты}
    \quad Введём две основные метрики, по которым будем сравнивать оба протокола:
    число сообщений, которые пришлось отправить отправителю, 
    и время работы протокола, за которое получатель смог получить все сообщения без повреждений.
    Посмотрим на зависимость этих метрик от 
    размера окна, времени таймаута и вероятности потери сообщения.
    
    Во всех тестах (если не сказано обратное) число сообщений, которые получатель должен получить от отправителя равно 100,
    а таймаут равен $ 0.5 $. Замеры времени работы проводились на CPU Intel I7-7700HQ 2.80GHz.

    Сначала посмотрим на зависимость числа сообщений и времени работы от размера таймаута.
    Размер окна равен 10, сообщение не может быть повреждено рис. \ref{p:timeoutsMessageNum} и рис. \ref{p:timeoutsWorkingTime}.

    \plot{timeoutsMessageNum}{Число сообщений от таймаута (размер окна = 10, вероятность повреждения сообщения = 0.0)}{p:timeoutsMessageNum}
    \plot{timeoutsWorkingTime}{Время работы от таймаута (размер окна = 10, вероятность повреждения сообщения = 0.0)}{p:timeoutsWorkingTime}

    Видно, что при очень малых значениях таймаута, отправитель для некоторых сообщений не успевает получить от получателя
    подтверждения до истечения времени ожидания, из-за чего посылает повторные сообщения.
    Но с увеличением времени таймаута такие случаи пропадают и число всех отправленных сообщений равно числу успешно переданных сообщений.

    Теперь рассмотрим значения числа всех отправленных сообщений и времени работы в зависимости от размера окна.
    Сначала для протокола Go-Back-N.

    \plot{rateSizeGBNMessageNum}{Go-Back-N. Число сообщений от размера окна}{p:rateSizeGBNMessageNum}
    \plot{rateSizeGBNWorkingTime}{Go-Back-N. Время работы от размера окна}{p:rateSizeGBNWorkingTime}

    Затем для протокола Selective Repeat.

    \plot{rateSizeSRPMessageNum}{Selective Repeat. Число сообщений от размера окна}{p:rateSizeSRPMessageNum}
    \plot{rateSizeSRPWorkingTime}{Selective Repeat. Время работы от размера окна}{p:rateSizeSRPWorkingTime}

    Как видно из рисунка \ref{p:rateSizeGBNMessageNum} общее число отправленных сообщений в протоколе Go-Back-N
    прямо пропорционально размеру окна, что особенно заметно при больших значениях вероятности потери сообщения.
    Но при этом время работы Go-Back-N не зависит от размера окна (рис. \ref{p:rateSizeGBNWorkingTime}).
    С другой стороны, для протокола Selective Repeat размер окна не влияет на общее число отпралвенных сообещний (рис. \ref{p:rateSizeSRPMessageNum}).
    При этом время работы протокола Selective Repeat тем меньше, чем больше размер окна,
    но с увеличением размера прирост скорости работы становиться всё меньше (рис. \ref{p:rateSizeSRPWorkingTime}). 

    Также рассмотрим зависимость тех же метрик от вероятности потери сообщения.
    Для Go-Back-N имеем.

    \plot{sizeRateGBNMessageNum}{Go-Back-N. Число сообщений от вероятности потери сообщения}{p:sizeRateGBNMessageNum}
    \plot{sizeRateGBNWorkingTime}{Go-Back-N. Время работы от вероятности потери сообщения}{p:sizeRateGBNWorkingTime}

    А для Selective Repeat.
    
    \plot{sizeRateSRPMessageNum}{Selective Repeat. Число сообщений от вероятности потери сообщения}{p:sizeRateSRPMessageNum}
    \plot{sizeRateSRPWorkingTime}{Selective Repeat. Время работы от вероятности потери сообщения}{p:sizeRateSRPWorkingTime}

    Как видно на рисунках \ref{p:sizeRateGBNMessageNum}, \ref{p:sizeRateSRPMessageNum} общее число отправленных сообщений с ростом вероятности потери сообщения
    у протокола Go-Back-N сильно больше, чем у протокола Selective Repeat. Как следствие, Go-Back-N работает значительно дольше, чем Selective Repeat (рис. \ref{p:sizeRateGBNWorkingTime}, \ref{p:sizeRateSRPWorkingTime}). 

    \section{Обсуждение}
    \quad Из приведённых результатов можно заметить, что в одинаковых условиях протоколу Selective Repeat требуется отправить меньше сообщений, чем протоколу Go-Back-N.
    Что ожидаемо, в силу разной обработки и повторной передачи потерянных сообщений (раздел \ref{s:theory}).
    Как следствие, протокол Selective Repeat работает значительно быстрее протокола Go-Back-N.

\end{document}