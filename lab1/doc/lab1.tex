\documentclass[a4paper,12pt]{article}

\usepackage[hidelinks]{hyperref}
\usepackage{amsmath}
\usepackage{mathtools}
\usepackage{shorttoc}
\usepackage{cmap}
\usepackage[T2A]{fontenc}
\usepackage[utf8]{inputenc}
\usepackage[english, russian]{babel}
\usepackage{xcolor}
\usepackage{graphicx}
\usepackage{float}
\graphicspath{{./images/}}

\definecolor{linkcolor}{HTML}{000000}
\definecolor{urlcolor}{HTML}{0085FF}
\hypersetup{pdfstartview=FitH,  linkcolor=linkcolor,urlcolor=urlcolor, colorlinks=true}

\DeclarePairedDelimiter{\floor}{\lfloor}{\rfloor}

\renewcommand*\contentsname{Содержание}


\begin{document}
    \begin{titlepage}
	\begin{center}
		{\large Санкт-Петербургский политехнический университет\\Петра Великого\\}
	\end{center}
	
	\begin{center}
		{\large Физико-механический иститут}
	\end{center}
	
	
	\begin{center}
		{\large Кафедра «Прикладная математика»}
	\end{center}
	
	\vspace{8em}
	
	\begin{center}
		{\bfseries Отчёт по лабораторной работе \\по дисциплине «Компьютерные сети» \\Задача византийских генералов }
	\end{center}
	
	\vspace{4em}
	
	\begin{flushleft}
		\hspace{16em}Выполнил студент:\\\hspace{16em}Куксенко Кирилл Сергеевич\\\hspace{16em}группа: 5040102/20201
		
		\vspace{2em}
		
		\hspace{16em}Проверил:\\\hspace{16em}к.ф.-м.н., доцент\\\hspace{16em}Баженов Александр Николаевич
		
	\end{flushleft}
	
	
	\vspace{6em}
	
	
	\begin{center}
		Санкт-Петербург\\2023 г.
	\end{center}	
	
\end{titlepage}
    \newpage

    \tableofcontents
    \newpage

    \section{Постановка задачи}

    \quad Нужно реализовать систему из двух объектов - отправителя (Sender) и получателя (Receiver),
    которые будут обменеваться сообщениями по каналу связи с помощью протоколов автоматического запроса повторений передачи
    с плавающим окном: Go-Back-N и Selective Repeat.

    Необходимо выяснить зависимость времени работы и количесвто посланных сообщений от размера плавающего окна 
    и вероятности потерисообщения для каждого протокола и сравнить друг с другом.

    \section{Реализация}

    \quad Весь код написан на языкe Python (версии 3.7.3).
    Для каждого протокола получатель и отправитель работают параллельно в отдельных потоках.
    

    \section{Результаты}

    \section{Обсуждение}

\end{document}